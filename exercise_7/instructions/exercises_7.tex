\documentclass[]{erlangen-problemset}
%\documentclass[a4paper]{scrartcl}
\usepackage{amsmath} %the recommended functionalities are align and gather for several equations (split allows arranging by hand; gather centers the equations), split for one equation over several lines (for both use '&' for the alignment); and multline for long expressions, which puts the first line left-aligned and the last line right-aligned! 
\usepackage[utf8]{inputenc}
\usepackage[english]{babel}
\usepackage{hyperref}
\usepackage{url}
\usepackage{listings}
\usepackage{graphicx}
\usepackage{amssymb}
\usepackage{color}
\usepackage{csquotes}
\usepackage{tensor} %e.g. \tensor[^a_b^c_d]{M}{^a_b^c_d}
%\usepackage{breqn} %better use a function from the amsmath package
\usepackage{easytable} %more options to produce tables!
%\usepackage[backend=biber,style=chem-angew,sorting=none, maxbibnames = 99]{biblatex}
\usepackage{bm} %fat text in formulas
\usepackage{afterpage}
\usepackage[toc,page]{appendix}
%\usepackage[usenames, dvipsnames]{color}
\usepackage{enumitem}
\usepackage{braket}
\usepackage{subfiles}
\usepackage{siunitx}

\newcommand{\del}{\partial}
\newcommand{\eqbox}[1]{\mbox{\boxed{#1}}}
\newcommand{\textitbf}[1]{\textbf{\textit{#1}}}
\newcommand{\red}[1]{\textcolor{red}{#1}}
\newcommand{\blue}[1]{\textcolor{blue}{#1}}
\newcommand{\green}[1]{\textcolor{green}{#1}}
\DeclareOldFontCommand{\bf}{\normalfont\bfseries}{\mathbf}
\newcommand{\op}[1]{\hat{\textbf{#1}}}
\renewcommand{\d}{\mathrm{d}}
\newcommand{\kb}{k_\text{B}}
\newcommand{\ev}[1]{\langle{#1}\rangle}


%Penalties
\widowpenalty10000
\clubpenalty10000

\setcounter{problemset}{7}

\title{{\Large Advanced Python for Research Projects} \\[0.3cm] 
Exercise sheet 7: An example project}

\begin{document}
%\maketitle 


\begin{problem}[title={Planning a project}]
\noindent
\Question 
\end{problem}


\begin{problem}[title={Working with other members}]
\noindent
\Question Set up a standard project on the university gitlab.
\Question Add another member.
\Question Come up with a small project (e.g. writing a simple many-body simulation with gravity)
\Question Plan the project in a way that profits from the multi-cores on modern computers. Plan for documentation and testing to occur. 
For this, identify key functionality that needs to be tested and identify interaction points with the user that need documentation.
\Question Agree on a git workflow (preferrably with feature branches)
\Question Work together to implement the parts necessary for your project on feature branches. Ask for feedback before merging through pull requests
\Question Use issues to keep track of what has been done and what needs to be done
\end{problem}

\begin{problem}[title={Automated testing}]
\noindent
\Question Write tests for key functionality of your code, e.g. writing the output to file, running simulations for a certain number of bodies, for how the input reading works, etc.
\Question Run these tests locally to confirm they are working
\Question Setup git to automatically run these tests (on gitlab)
\Question Test on the university gitlab, that the tests are run and inspect the results
\Question Open issues for all aspects that need fixing and fix them with a hotfix branch. 
\end{problem}

%\begin{problem}[title={Parallelization in python}]
%\noindent
%\Question Write function to create a file with a configurable number of random float32 entries to a parameter filename
%\Question Write function to read entries from a filename and calculate average
%\Question now write a function that writes N files with M entries, then reads their averages back and calculates a final average
%\Question Write a version of the function using a threadpool, where you use all cores of your CPU to parallalize the writing of files, then wait for that operation to finish, measure the time it took and then read them back with average calculation in parallel and measure the time for that as well. 
%Overall compare the times for the writing operation and the times for the read+calculation operation to the non-parallelized version. Which part scales better with the number of cores? And why?
%\Question now use the multiprocessing module to perform the part of the parallelized task that did not scale well with the number of cores.
%\Question How does the multiprocessing module perform in place of the threaded version?
%\end{problem}

\begin{problem}[title={Creating a release}]
\noindent
\Question Make sure that your code contains installation, usage and example information before creating the release
\Question Now that you are confident in your code performing the way it should, merge features and hotfixes into your main branch and create a release
\Question Add a useful explanation to your release.
\end{problem}

\begin{problem}[title={FAIR principles}]
\noindent
\Question Download the release zip from the git provider and import it into ZENODO to create a documented archive
\Question Add all necessary meta-information for your Project to be well-documented, including contributors, Usage-Information, a link to the git repository, etc. 
\Question publish your results and retrieve the DOI for your release
\end{problem}

\begin{problem}[title={Publish the package on PyPi}]
\noindent
\Question Perform the steps for a dummy-publication of your package on PyPi
\end{problem}

\end{document}
