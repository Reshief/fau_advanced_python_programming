\documentclass[]{erlangen-problemset}
%\documentclass[a4paper]{scrartcl}
\usepackage{amsmath} %the recommended functionalities are align and gather for several equations (split allows arranging by hand; gather centers the equations), split for one equation over several lines (for both use '&' for the alignment); and multline for long expressions, which puts the first line left-aligned and the last line right-aligned! 
\usepackage[utf8]{inputenc}
\usepackage[english]{babel}
\usepackage{hyperref}
\usepackage{url}
\usepackage{listings}
\usepackage{graphicx}
\usepackage{amssymb}
\usepackage{color}
\usepackage{csquotes}
\usepackage{tensor} %e.g. \tensor[^a_b^c_d]{M}{^a_b^c_d}
%\usepackage{breqn} %better use a function from the amsmath package
\usepackage{easytable} %more options to produce tables!
%\usepackage[backend=biber,style=chem-angew,sorting=none, maxbibnames = 99]{biblatex}
\usepackage{bm} %fat text in formulas
\usepackage{afterpage}
\usepackage[toc,page]{appendix}
%\usepackage[usenames, dvipsnames]{color}
\usepackage{enumitem}
\usepackage{braket}
\usepackage{subfiles}
\usepackage{siunitx}

\newcommand{\del}{\partial}
\newcommand{\eqbox}[1]{\mbox{\boxed{#1}}}
\newcommand{\textitbf}[1]{\textbf{\textit{#1}}}
\newcommand{\red}[1]{\textcolor{red}{#1}}
\newcommand{\blue}[1]{\textcolor{blue}{#1}}
\newcommand{\green}[1]{\textcolor{green}{#1}}
\DeclareOldFontCommand{\bf}{\normalfont\bfseries}{\mathbf}
\newcommand{\op}[1]{\hat{\textbf{#1}}}
\renewcommand{\d}{\mathrm{d}}
\newcommand{\kb}{k_\text{B}}
\newcommand{\ev}[1]{\langle{#1}\rangle}


%Penalties
\widowpenalty10000
\clubpenalty10000

\setcounter{problemset}{3}

\title{{\Large Advanced Python for Research Projects} \\[0.3cm] 
Exercise sheet 4: Modules, Documentation and testing}

\begin{document}
%\maketitle 


\begin{problem}[title={Using files and directories as modules}]
\noindent
\Question Pack previously defined functions in a file called advanced_functions.
\Question Add documentation to the file like for a module 
\Question Add a function to output the module name using \_\_name\_\_
\Question Create a directory advanced\_module and a file called \_\_init\_\_.py in there
\Question Print out the module name \_\_name\_\_ in the \_\_init\_\_.py script and create two more files arithmetics.py and func.py. 
Add functions to get the square root of the absolute value of a number as well as the function abs\_sub, that takes the absolute value of the difference of the abs of its two parameters to arithmetics.py, add documentation. 
Add functions to load the first two lines and the last two lines from a file in func.py
\Question Create a main script main.py and import the module advanced_functions and the module advanced_module. 
What do you notice when you execute the script? 
\Question Use the function from the advanced\_functions module to output the module name, which module name does it print compared to a print(\_\_name\_\_) statement in main.py?
\Question Add loggers to your modules using the \_\_name\_\_ variable to determine the module name and add them to your module functions to log infos, warnings and errors
\Question Use the functions from the other modules to read the first lines from the file mystic.txt and calculate the abs\_sub value of the numbers in the first two lines and calculate the abs square roots for the numbers in the last two lines. Output your values using a logger in your main file. 
\Question Add some function that can trigger an error state and test that it logs correctly. 
\Question Add the map and the reduce function to their own helper-module, document it, add tests and generate a documentation file. (look up how to do that)
\end{problem}


\begin{problem}[title={Building tests for our modules}]
\noindent
\Question Add tests for your functions to check that they work, also for the modules defined via files or as a directory
\Question Run the tests and verify correct operation. Fix any errors you notice and Add comments explaining the intention of your tests
\end{problem}

\begin{problem}[title={Making our module installable with pip}]
\noindent
\Question We want to use pip -e . to install our module advanced\_module so that any python script on our user can access it. (Link to example and documentation)
\Question Install the module to the user
\Question Remove the module from the user
\Question Add a Readme.md file explaining the function of your module.
\end{problem}

\begin{problem}[title={Generating documentation for your module}]
\noindent
\Question Use pdoc3 to generate a documentation page for your module
\end{problem}


\end{document}
