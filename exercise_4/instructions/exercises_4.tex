\documentclass[]{erlangen-problemset}
%\documentclass[a4paper]{scrartcl}
\usepackage{amsmath} %the recommended functionalities are align and gather for several equations (split allows arranging by hand; gather centers the equations), split for one equation over several lines (for both use '&' for the alignment); and multline for long expressions, which puts the first line left-aligned and the last line right-aligned! 
\usepackage[utf8]{inputenc}
\usepackage[english]{babel}
\usepackage{hyperref}
\usepackage{url}
\usepackage{listings}
\usepackage{graphicx}
\usepackage{amssymb}
\usepackage{color}
\usepackage{csquotes}
\usepackage{tensor} %e.g. \tensor[^a_b^c_d]{M}{^a_b^c_d}
%\usepackage{breqn} %better use a function from the amsmath package
\usepackage{easytable} %more options to produce tables!
%\usepackage[backend=biber,style=chem-angew,sorting=none, maxbibnames = 99]{biblatex}
\usepackage{bm} %fat text in formulas
\usepackage{afterpage}
\usepackage[toc,page]{appendix}
%\usepackage[usenames, dvipsnames]{color}
\usepackage{enumitem}
\usepackage{braket}
\usepackage{subfiles}
\usepackage{siunitx}

\newcommand{\del}{\partial}
\newcommand{\eqbox}[1]{\mbox{\boxed{#1}}}
\newcommand{\textitbf}[1]{\textbf{\textit{#1}}}
\newcommand{\red}[1]{\textcolor{red}{#1}}
\newcommand{\blue}[1]{\textcolor{blue}{#1}}
\newcommand{\green}[1]{\textcolor{green}{#1}}
\DeclareOldFontCommand{\bf}{\normalfont\bfseries}{\mathbf}
\newcommand{\op}[1]{\hat{\textbf{#1}}}
\renewcommand{\d}{\mathrm{d}}
\newcommand{\kb}{k_\text{B}}
\newcommand{\ev}[1]{\langle{#1}\rangle}


%Penalties
\widowpenalty10000
\clubpenalty10000

\setcounter{problemset}{4}

\title{{\Large Advanced Python for Research Projects} \\[0.3cm] 
Exercise sheet 4: Scientific computation}

\begin{document}
%\maketitle 


\begin{problem}[title={Using numpy and scipy for scientific calculation}]
\noindent
\Question Generating random numbers
\Question Operations with indices
\Question Writing and reading files with numpy
\Question Calculating statistics
\Question Visualizing statistics
\Question Other Scipy modules
\end{problem}

\begin{problem}[title={Rebuilding numpy and scipy functionality ourselves}]
\noindent
\Question Differentiation
\Question Integration
\Question Add Documentation
\Question Add Tests and execute them to ensure correct functionality
\end{problem}

\begin{problem}[title={Parallel numpy/scipy replacement}]
\noindent
\Question Let us employ our parallellized map and reduce functions to parallelize our numpy/scipy imitation functions from the prior exercise
\Question add tests to make sure, the operation is identical to the result obtained by numpy/scipy.
\Question Compare their performance to our previous sequential versions as well to the numpy/scipy implementation
\Question What do we learn from the comparison in performance?
\end{problem}

\begin{problem}[title={Measuring performance}]
\noindent
\Question Write a function to compare the speed of numpy and scipy functions to our own implementation
\Question Plot how the execution time of various functions behaves as a function of the input size
\Question What is the conclusion?
\end{problem}

\end{document}
